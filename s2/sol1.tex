En utilisant l'astuce, on montre directement que cette définition forme une norme matricielle.

\begin{enumerate}[label=\alph*)]
  \item Si $\norm{A \BoldX} \geq 0$, alors $\norm{A} = \sup_{\norm{\BoldX} = 1} \norm{A \BoldX} \geq 0$. De plus
  
  \begin{equation*}
    \norm{A} = \sup_{\BoldX \neq \BoldZero} \dfrac{\norm{A \BoldX}}{\BoldX} = 0
    \LRA
    \norm{A \BoldX} = 0 \ \forall \ \BoldX \neq \BoldZero,
  \end{equation*}
  
  ou encore
  
  \begin{equation*}
    A \BoldX = 0 \ \forall \ \BoldX \neq \BoldZero
    \LRA
    A = \BoldZero.
  \end{equation*}
  
  Donc, $\norm{A} = 0 \LRA A = \BoldZero$.
  
  \item Soit un scalaire $\alpha$, on a 
  
  \begin{equation*}
    \norm{\alpha A}
    = \sup{\norm{\BoldX} = 1} \norm{\alpha A \BoldX}
    = \abs{\alpha} \sup{\norm{\BoldX} = 1} \norm{A \BoldX}
    = \abs{\alpha} \norm{A}.
  \end{equation*}
  
  \item Vérifions enfin l'inégalité triangulaire. Par définition du supremum, si $\BoldX \neq \BoldZero$ alors
  
  \begin{equation*}
    \dfrac{\norm{A \BoldX}}{\BoldX} \leq \norm{A}
    \Rightarrow \norm{A \BoldX} \leq \norm{A} \norm{\BoldX},
  \end{equation*}
  
  ainsi, en prenant $\BoldX$ de norme 1, on obtient
  
  \begin{equation*}
    \norm{\parent{A + B} \BoldX}
    \leq \norm{A \BoldX} + \norm{B \BoldX}
    \leq \norm{A} + \norm{B}, 
  \end{equation*}
  
  d'où on déduit $\norm{A + B} = \sup_{\norm{\BoldX} = 1} \norm{\parent{A + B} \BoldX} \leq \norm{A} + \norm{B}$.
  
  
  
\end{enumerate}



