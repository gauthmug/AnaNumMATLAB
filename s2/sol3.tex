Si $\lambda$ est une valeur propre de $A$, alors il existe $\BoldV \neq \BoldZero$, vecteur propre de $A$, tel que $A \BoldV = \lambda \BoldV$.
Ainsi, puisque $\norm{\cdot}$ est consistante,

\begin{equation*}
  \abs{\lambda} \norm{\BoldV}
  = \norm{\lambda \BoldV}
  = \norm{A \BoldV}
  \leq \norm{A} \norm{\BoldV}
\end{equation*}

et donc $\abs{\lambda} \leq \norm{A}$.
Cette inégalité étant vraie pour toute valeur propre de $A$, elle l'est en particulier quand $\abs{\lambda}$ est égale au rayon spectral.