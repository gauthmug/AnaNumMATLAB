\begin{center}
  \textbf{Exercices Matlab - Analyse Numérique - 2017 \\
  Section MA \\
  Prof. A. Quarteroni \\
  Séance 1 - Introduction à Matlab}
\end{center}


\vspace{10mm}

\textbf{Exercice 1 \\}
On considère les matrices
\begin{equation*}
  A = \begin{bmatrix}
        5 & 3 & 0  \\
        1 & 1 & -4 \\
        3 & 0 & 0
      \end{bmatrix}
  ,
  B = \begin{bmatrix}
        4 & 3 & 2  \\
        0 & 1 & 0  \\
        5 & 0 & 1/2
      \end{bmatrix}
\end{equation*}

Créer un répertoire de travail et écrire un fichier ".m" dans lequel placer les instructions pour calculer (sans utiliser de boucles), la matrice $C = AB$ (produit matriciel) et la matrice $D$ qui a comme éléments $D_{ij} = A_{ij} B_{ij}$ (produit composante par composante).


\lstinputlisting{matlab/ex1.m}





\textbf{Exercice 2 \\}
Définir (sans utiliser de boucles) la matrice diagonale de taille $n = 5$ dont la diagonale est un vecteur de points équirépartis entre $3$ et $6$ (i.e. $\squared{3, 3.75, 4.5, 5.25, 6}$).